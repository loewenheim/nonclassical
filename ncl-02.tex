\documentclass[handout]{beamer}
%\documentclass{beamer}
\usepackage[latin1]{inputenc}
\usepackage{proof}
%\usepackage{latexsym}
\usepackage{amssymb}
\usepackage{gslist}
\usepackage{gsgraph}
%\usepackage{pstricks}
\usepackage{ncl}

\def\N{{\mathbb N}}
\def\ee{\doteq}
\def\lee{\lessdot}
\definecolor{darkgreen}{rgb}{0,.3,0}
\definecolor{darkred}{rgb}{.5,0,0}
\definecolor{darkblue}{rgb}{0,0,.7}
\definecolor{darkvio}{rgb}{.5,0,.5}
\definecolor{itemblue}{rgb}{0,0,.4}

\renewcommand{\emph}[1]{{\color{blue} #1}}
\newcommand{\empha}[1]{{\color{darkgreen} #1}}
\newcommand{\emphb}[1]{{\color{darkvio} #1}}
\renewcommand{\em}[1]{{\color{darkblue} #1}}
\newcommand{\deem}[1]{{\color{itemblue} #1}}

\newcommand{\headline}[1]{{\color{blue} \Large\sffamily\bfseries#1}\par}
\newcommand{\smallhead}[1]{{\color{blue} \large\sffamily\bfseries#1}}
\newcommand\para[1]{{\sffamily\bfseries#1}}
%\newcommand\labelitemi{\mbox{$\diamond$}}
\newcounter{exn}
\begin{document}

\setcounter{exn}{10}


\title{\em{\huge Nonclassical logics}\\
      \em{\huge = `Modal and Epistemic Logic'}\\
       VU 185.249 (lecture + exercises)}
\author{\fbox{\tt http://www.logic.at/lvas/ncl/}}
\institute{Chris Ferm\"uller\\
  Technische Universit\"at Wien\\
  \texttt{www.logic.at/people/chrisf/}
}
\date{Winter term 2015/16\\[2ex]
\empha{Lecture 2}}
\frame<1-| handout:0>{\titlepage}
%%%%%%%%%%%%%%%

\frame{

\headline{Modal logics}

\empha{Remark:}\\
Like CL (classical logic), modal logics comes in \em{levels/orders}.\\
We will concentrate on \em{propositional} modal logics.

\vspace{1ex}
\pause
\deem{Syntax:}
 \bit
  \item additional unary connective(s):\\
        \em{modal operator}~\n{} (`box')\\
        and the dual operator~\p{} (`diamond'):\ \ 
        $ \p =_{def} \neg \n \neg $
\pause
  \item inductive definition of formulas ($\FORM$) as usual.
\eit

\pause
Later we will extend this to \em{multi-modal logics}:
\bit
\pause
\item E.g.,  $K_1, K_2, \ldots, K_n, S_G, D_G, E_G$ 
  will all serve as (unary) \em{modal operators} of a 
  single \em{epistemic logic}
\pause
\item other important multi-modal logics: 
    \bit 
    \item \em{temporal logics:} (unary and binary) operators 
           refer to past/future (time points)
%\pause
    \item \em{dynamic logic:} each program is a modal
          operator!
    \eit
\eit 
}

%%%%%%%%%%%%%
\frame{
\smallhead{Intended semantics}\\[0.5ex]

\ul{Modal} operators refer to a \em{\ul{mode} of assertion}:\\[1ex]

\pause

\begin{tabular}{|p{30ex}|p{30ex}|}
\hline
\empha{Meaning of $\n F$}      &     \empha{Meaning of $\p F (=\neg\n\neg F) $} \\[0.45ex]
\hline\hline
\em{necessarily} $F$            &  \pause  \em{possibly} $F$ \\[0.45ex]
\hline  \pause 
\em{always} $F$     &  \pause  \em{sometimes}~$F$  \\[0.45ex]
\hline  \pause 
$F$ \em{should be} the case      &  \pause  $F$ is \em{permitted}\\[0.45ex]
\hline \pause 

(an agent) \em{knows} that $F$      &  \pause (an agent) deems $F$ \em{consistent with what is known} to 
                                     her/him\\[0.45ex]
\hline  \pause 
(an agent) \em{believes} that $F$    &  \pause  (an agent) \em{deems} $F$ \em{possible}\\[0.45ex]
\hline  \pause 
$F$ is \em{provable}               &  \pause $F$ is \em{not refutable}\\[0.45ex]
\hline  \pause 
$F$ holds \em{at every  terminating state} of
a process/program  &  \pause $F$ holds \em{at some terminating state} of a
process/program \\[0.45ex]
\hline
\end{tabular}
}

%%%%%%%%%%%%%%%%%%
\frame{
\smallhead{Possible world semantics}\\[1ex]

\empha{Example:} Reasoning about databases\\
\bit
\item Formulas $F$ are built up from atoms $P(t_1, \ldots, t_n)$,
      where \em{ground atoms}  are \em{entries in a database}.
\pause
\item CL-formulas $F$ thus correspond to \em{queries},
      but may also be used to formulate \em{integrity constraints},
      i.e., formulas that have to hold in \em{every instance} of
      a database.
\pause
\item To \em{express} that $F$ is an \em{integrity constraint} 
      (and not just a formula holding in the current instance
      of the database, or just a valid formula) we have
      to extend CL:\\
      $\n F$ \dots $F$ is an integrity constraint
\pause
\item To extend (Tarski's) semantics for CL to the evaluation
      of statements like $\n F$, we have to refer to all
      \em{possible states} of a database.
\eit

\pause

\empha{Similar examples:}\\
(physically) necessary $\longrightarrow$ 
   (physically) possible states of the world\\
always $\longrightarrow$ system states at arbitrary time points\\
\pause
\em{etc. [check previous slide!]}

}
%%%%%%%%%%%%%%%
%%%%%%%% 2nd COPY - NOT ON HANDOUT
\frame<1-| handout:0>{
\smallhead{Intended semantics}\\[1ex]

Modal operators refer to a \em{mode of assertion}:\\[1ex]

\begin{tabular}{|p{30ex}|p{30ex}|}
\hline
\empha{Meaning of $\n F$}      &     \empha{Meaning of $\p F (=\neg\n\neg F) $} \\[0.45ex]
\hline\hline
\em{necessarily} $F$            &  \em{possibly} $F$ \\[0.45ex]
\hline 
\em{always} $F$     &  \em{sometimes}~$F$  \\[0.45ex]
\hline  
$F$ \em{should be} the case      &   $F$ is \em{permitted}\\[0.45ex]
\hline 
(an agent) \em{knows} that $F$      &   (an agent) deems $F$ \em{consistent with what is known} to 
                                     her/him\\[0.45ex]
\hline  
(an agent) \em{believes} that $F$    &  (an agent) \em{deems} $F$ \em{possible}\\[0.45ex]
\hline  
$F$ is \em{provable}               &  $F$ is \em{not refutable}\\[0.45ex]
\hline  
$F$ holds \em{at every  terminating state} of
a process/program  &   $F$ holds \em{at some terminating state} of a
process/program \\[0.45ex]
\hline
\end{tabular}
}
%%%%%%%% 
%%%%%%%%%%%%%%%

\frame{
\smallhead{(Formal) possible world semantics:\\ \pause
           Kripke semantics}\\[1ex]
\pause
\empha{Note:} We only treat \em{propositional} modal logics, here.\\[1ex]
\pause

A \em{Kripke interpretation (model)} is a tuple $\M = \langle W,R,V\rangle$
where:
\pause
\begin{itemize}
\item non-empty set $W$ of \em{(possible) worlds} (states, points)
\pause
\item an \em{accessibility relation} $R \subseteq W \times W$
\pause 
\item \em{(variable) assignment} $V : (\PV \times W) \mapsto \{\t,\f\}$\\
      (alternatively:  $V': \PV \mapsto 2^W$ or $V'':W \mapsto 2^{\PV}$)
\end{itemize}

\pause
Formulas without $\n$ are evaluated as usual (\em{in each world}), e.g.:
\[
v_\M(F \And G,w) = \left\{ \begin{array}{r@{\quad}l}
                        \t  & \mbox{ if } v_\M(F,w) = \t \\ 
                            & \mbox{ and } v_\M(G,w) = \t  \\
                        \f  & \mbox{ otherwise } 
                            \end{array}
                   \right.
\]
\pause
\empha{New:}\\[-4ex]
\[
v_\M(\n F,w) = \left\{ \begin{array}{r@{\quad}l}
                        \t  & \mbox{ if } \forall u\, wRu
                                          \Rightarrow v_\M(F,u) = \t \\ 
                        \f  & \mbox{ otherwise } 
                            \end{array}
                   \right.
\]
}
%%%%%%%%%%%%%%%
\frame{
\smallhead{Kripke semantics (ctd.) }\\[1ex]

Since $\p = \neg \n \neg$ we obtain:\\[-2ex]
\[
v_\M(\p A,w) = \left\{ \begin{array}{r@{\quad}l}
                        \t  & \mbox{ if } \exists u\, wRu
                                          \mbox{ and } v_\M(A,u) = \t \\ 
                        \f  & \mbox{ otherwise } 
                            \end{array}
                   \right.
\]
\pause
\UEl{Derive the above (in detail) from  $\p = \neg \n \neg$
and from  the defining\\ conditions of the semantics for $\n F$ and for $\neg F$.}

\vspace*{1ex}

\pause
\empha{Alternative  notation} for $ v_{\M}(A,w) = \t $:\
 \ $ (\M,w) \models A$\\
\em{Read:} `$A$ holds at state/point/world $w$ (in $\M$)'\\
\hspace*{3ex}\em{or:} `$A$ is satisfied at state/point/world $w$ (in $\M$)'\\
\hspace*{3ex}\em{or:} `$A$ is forced at state/point/world $w$ (in $\M$)'\\
\hspace*{3ex}\em{or:} `$A$ is true at state/point/world $w$ (in $\M$)'\\[1ex]
\pause

\pause
$A$ is \em{true in interpretation (model)}
 $\M = \langle W,R,V\rangle$  if $v_\M(A,w) = \t$ for all $w \in W$.\\
\empha{Notation:} $\M \models A$\\[1ex]

\pause
The pair  $\langle W,R \rangle$ of an interpretation
 $\M = \langle W,R,V\rangle$\\ is called the \em{(Kripke) frame} 
on which~$\M$ is \em{based}.

}
%%%%%%%%%%%%%%%
\frame{
\smallhead{\ul{Four} levels of `truth' in Kripke semantics}

\ben
\item \em{Truth at a world:} $ (\M,w) \models A$
\pause
\item \em{Truth in a model:} $\M \models A$.
\pause
\item \em{Validity (truth) in a frame:} \\
$A$ is \em{valid in frame}
 $\F = \langle W,R\rangle$  if $A$ is true in all interpretations
based on $\F$.\\
 \empha{Notation:} $\F \models A$
\pause
\item \em{Validity (truth) in a class of frames:}\\
  If $\Psi$ is a class (set) of frames we say that $A$ is
  \em{true in $\Psi$} if $\F \models A$ for all $\F \in \Psi$.
\een

\empha{Note:}\\
A modal \em{logic} does not refer to a particular frame,
but to a \em{class of frames} determined by some \em{property of
the accessibility relations}.

}
%%%%%%%%%%%%%%%
\frame{
\smallhead{Evaluation in a Kripke interpretation --- Example}\\[1ex]

$\M = \langle \{0,1,2,3,4\},R,V\rangle$:
\renewcommand\ggrid{4000}
\renewcommand\gunitlength{0.0045mm}
%\fbox{%
\hspace*{10ex}\begin{graph}(4.2,2)(-0.5,-1.5)
\pos  c=({0.5},-1);d=({0.5},{-0.5}).
\pos  0=(0,0);1=(1,0);2=(1,-1);3=(2,0);4=(3,0).
%\style edgelabel{close}.
\style node{plain};nodelabel{0};edgelabel{W}.
\node 0{0};1{1};2{2};3{3};4{4}.
\style nodelabel{SE}.
\node 0{$q$};1{$p$};2{$p,q$};3{$q$};4{$p,q$}.
\edge 0-1-2-c-d-1-3-3-4.
\end{graph}%

Evaluate the following formulas in different worlds in $\M$:\\[1ex]
1. $p \Impl q$, $q \Impl p$, $p \Or q$, $p \Or \neg p$\\
2. $\p p$, $\n p$, $\p q$, $\n q$, $\p \neg q$,  $\n \neg q$\\
3. $\n (p \Or q)$, $\p (p \Or q)$, $\p (p \Impl q)$, 
   $\p (p \And \neg p)$, $\n (p \Or \neg p)$\\
4. $\n p \Or \n \neg p$, $\n p \Impl \p p$, $\neg \p p \Impl \n p$, 
   $\n q \Impl q$, $p \Or \p p$\\
5. $\p \n p$, $\n \p q$, $\p \p p$, $\n \n p \Impl \n p$, 
         $\n \n q \Impl \n q$, $\n p \Impl \n \n p$ \\
6. $\p(p \And q) \Impl \p p$, $\p \neg q \Impl \neg \n q$,
   $\n (p \And q) \Impl (\n p \And \n q)$
}
%%%%%%%%%%%%%%%
\frame{
\UEl{
For the formulas of lines 1--6 of the last slide:\\
In which worlds of  the following 
interpretation\\
$\M = \langle \{0,1,2,3\},R,V\rangle$ is the corresponding formula true?\\
\renewcommand\ggrid{6000}
\renewcommand\gunitlength{0.0045mm}
%\fbox{%
\hspace*{4ex}\begin{graph}(4.2,2.4)(-0.5,-1.5)
\pos  c=(0,{0.5});d=({2},{0.5}).
\pos  0=(0,0);1=(1,0);2=(1,-1);3=(2,0).
%\style edgelabel{close}.
\style node{plain};nodelabel{0};edgelabel{W}.
\node 0{0};1{1};2{2};3{3}.
\style nodelabel{SE}.
\node 1{$q$};2{$p,q$};3{$p,q$}.
\edge 0-c-d-3.
\edge 0-1-3.
\edge 1-1-2.
\edge 0-2.
\end{graph}%
}
}
%%%%%%%
\frame{

\em{The following homework is compulsory:}

\vspace{1ex}
   
\UEl{For each of the following formulas:\\
Find a Kripke model in which the formula
is true in some world. Is there a  Kripke model for the formula?
% (I.e., can it be true in all worlds?)  
If not, why? \\ (Try to use as few worlds as possible.)
\bit
\item $\p p \And \p \n p \And \neg \n p$ 
\item $p \And \n p \And \neg \p p$
\item $(p \Impl q) \And \p (p \And \neg q)$
\item $\neg p \And \p\p p \And \neg \n\p p \And \p\n \neg p$
\item $\p p \And (\n p \Impl \n\p p) \And \n(p \Impl \neg \p p)$ 
\eit
}}
%%%%%%%%%%%%%
\frame<1-| handout:0>{
\smallhead{\ul{Four} levels of `truth' in Kripke semantics}

\ben
\item \em{Truth at a world:} $ (\M,w) \models A$
\item \em{Truth in a model:} $\M \models A$.
\item \em{Validity (truth) in a frame:} \\
$A$ is \em{valid in frame}
 $\F = \langle W,R\rangle$  if $A$ is true in all interpretations
based on $\F$.\\
 \empha{Notation:} $\F \models A$
\item \em{Validity (truth) in a class of frames:}\\
  If $\Psi$ is a class (set) of frames we say that $A$ is
  \em{true in $\Psi$} if $\F \models A$ for all $\F \in \Psi$.
\een

\empha{Note:}\\
A modal \em{logic} does not refer to a particular frame,
but to a \em{class of frames} determined by some \em{property of
the accessibility relations}.

}

%%%%%%%%%%%%%%%

\frame{
\smallhead{Validity in frames}\\[1ex]

Consider the frame $\F = \langle \{w,u\},R\rangle$, where $R=W \times W$:\\
\renewcommand\ggrid{6000}
\renewcommand\gunitlength{0.0045mm}

\hspace*{4ex}
\begin{graph}(2,0.5)(0,0)
\pos  0=(0,0);1=(1,0).
%\style edgelabel{close}.
\style node{plain};nodelabel{0}.
\node 0{w};1{u}.
\edge 0-0-1-1-0.
\end{graph}
\vspace{2ex}

The following formulas (\em{schemata}) are valid in $\F$: \pause
\bit
\item $A \Impl \p A$ \pause --- not valid without $R(w,w)$ or $R(u,u)$. 
\pause
\item $\p (A \Impl  A)$ \pause 
--- not valid without, e.g., $R(w,w)$ and $R(w,u)$.
\eit
\vspace{0.5ex}

\em{Compulsory homework:}\\[0.5ex]
\pause
\UEl{Find 3-5 further examples of modal formulas with one schematic
variable that are valid in $\F$, above, such that removal of some (which?)
accessibilities leads to invalidity.}
\pause

\vspace{0.5ex}
\empha{Note:}\\
\em{Modal logics} usually do not refer to just a particular frame,
but to whole  \em{classes of frames}, that are determined by  \em{properties of
the accessibility relation}.

}
%%%%%%%%


\end{document}

