\documentclass[a4paper,10pt]{article}
\usepackage[utf8]{inputenc}
\usepackage{uniinput}
\usepackage{amsthm}
\usepackage{amsmath}
\usepackage{amssymb}
\usepackage{fullpage}

\newcommand{\imp}{\supset}
\newcommand{\I}{\mathcal{I}}
\newcommand{\Ib}{\I_{¬ F}}
\newcommand{\It}{\I_{F}}
\renewcommand{\L}{\mathcal{L}}
\newcommand{\Lf}{\L_F}


\newtheorem{theorem}{Theorem}
\newtheorem{lemma}[theorem]{Lemma}
\newtheorem*{problem*}{Problem}

\theoremstyle{definition}

\newtheorem{definition}[theorem]{Definition}

\begin{document}
\section{Exercise 1}
\begin{problem*}
Explain (in your own words!) the paradoxes of material implication.
\end{problem*}
Paradoxes of material implication—at least some of them—result from the fact that the truth of $p \imp q$ is defined solely via the individual truth values of $p$ and $q$ and not any causal or meaningful relationship between them. For instance, since $p \imp q$ is true if $q$ is true, a true proposition is implied by everything, which is counterintuitive. Conversely, a false proposition implies everything, which is just as paradoxical. Combining these two one obtains the tautology $(p \imp q) ∨ (q \imp p)$. If $p$ is false, the first disjunct is true and if $p$ is true, the second one is true. But in natural language, it would be nonsensical to say that between two statements, one always implies the other. The problem is that expressions like ``implies'' or ``if ... then'' carry connotations, like causality and relevance, that material implication does not capture.

\section{Exercise 2}
\begin{problem*}
Consider the following statement
$S$, that we will assume to be true:
\emph{If god doesn't exist then it is not the case that if I pray to god she will answer my prayers.}
This  has  the  logical  form $¬ G \imp ¬(P \imp A)$. I  don't  pray  to  god means  that $P$ is  false.   But  if $P$ is  false  then
$P \imp A$ is  true  and therefore $¬(P \imp A)$  is  false.   Since  we  assume $S$ to  be  true, $¬ G$ has also to be false and thus
$G$ (`God exists') is true.
What's  wrong  here?   In  other  words:  Why  can't  one  prove  god's existence from the (innocent) statement
$S$ by not praying?
\end{problem*}

First of all, $S$ can be restated as $(P \imp A) \imp G$ (``If it is true that if I pray to God, then God anwers, then God exists.'') by contraposition. The problem here is that the premise of the outer implication, $P \imp A$, is intended to capture the notion ``God answers my prayers'', or something like it, but it does not precisely achieve that. The two statements diverge in the case of $¬P$, which is of course exactly the problematic case: $(P \imp A)$ becomes true, whereas ``God answers my prayers'' becomes technically true, but essentially meaningless (or even deceptive, depending on the intent of the speaker).

Another way to restate the problem: Suppose I want to convince you that God exists by saying ``God has answered all my prayers, therefore she exists''. This argument seems reasonable on the face of it, but God answering all prayers I say only carries weight if I actually say any. If I've never prayed, this statement remains true, but only vacuously so: hearing me say this sentence, you assume that there is at least one prayer that I have said and God has therefore answered.
\section{Exercise 4}
\begin{problem*}
Show  that  the  satisfiability  problem  for  the  fragment  of  propositional CL without $¬$ and $\bot$ (``positive fragment'') is trivial, whereas the validity problem is still coNP-complete.
\end{problem*}

Concerning satisfiability: every positive formula is satisfiable by evaluating all propositional atoms as true. This is easily seen via induction.

The rest of this section deals with validity.

Let $\L$ be a fixed propositional language, i.e. a set of propositional atoms, that does not contain the symbol $F$. Let $\Lf = \L ∪ \{F\}$.
\begin{definition}
 Let $φ$ a propositional formula in the language $\L$. We define a translation $'$ into the positive fragment of propositional logic:
 \begin{enumerate}
  \item For $φ \equiv A$: $φ' \equiv A$.
  \item For $φ \equiv \bot$: $φ' \equiv F$.
  \item For $\circ \in \{∨,∧,\imp\}$ and $φ \equiv ψ_1 \circ ψ_2$: $φ' \equiv ψ_1' \circ ψ_2'$.
  \item For $φ \equiv ¬ψ$: $φ' \equiv ψ' \imp F$.
 \end{enumerate}
\end{definition}

Clearly, the range of $'$ contains no instances of $¬$ or $\bot$.

If $\I$ is an interpretation of $\L$, let $\Ib = \I ∪ \{ F \mapsto \bot\}$ and $\It = \I ∪ \{F \mapsto \top\}$ be interpretations of $\Lf$.
\begin{lemma}
 Let $\I$ be an interpretation of $\L$ and $φ$ an $\L$-formula.
 \begin{enumerate}
  \item If $\I \vDash φ$ then $\Ib \vDash φ'$ and $\It \vDash φ'$.
  \item If $\I \not\vDash φ$ then $\Ib \vDash φ' \imp F$.
 \end{enumerate}
\end{lemma}
\begin{proof}
 By induction.
 \begin{itemize}
  \item $φ \equiv A$: trivial.
  \item $φ \equiv \bot$: trivial.
  \item $φ \equiv ψ_1 ∧ ψ_2$: Assume $\I \vDash φ$. Then $\I \vDash ψ_1$ and $\I \vDash ψ_2$. By induction, $\Ib \vDash ψ_i'$ and $\It \vDash ψ_i'$ for $i \in \{1,2\}$ and hence $\Ib \vDash ψ_1' ∧ ψ_2 \equiv φ' $ and $\It \vDash ψ_1' ∧ ψ_2' \equiv φ'$.
  
  Assume $\I \not \vDash φ$. Then $\I \not \vDash ψ_1$ or $\I \not \vDash ψ_2$. Say $\I \not \vDash ψ_1$ w.l.o.g. Then by induction, $\Ib \vDash ψ_1' \imp F$ and hence $\Ib \vDash (ψ_1' ∧ ψ_2') \imp F \equiv φ' \imp F$.
  \item $φ \equiv ψ_1 ∨ ψ_2, φ \equiv ψ_1 \imp ψ_2$: Analogous to the previous case.
  \item $φ \equiv ¬ ψ$: Assume $\I \vDash φ$, then $\I \not \vDash ψ$. By induction $\Ib \vDash ψ' \imp F \equiv φ'$. $\It \vDash ψ' \imp F \equiv φ'$ is trivially true.
  
  On the other hand, if $\I \not \vDash φ$, then $\I \vDash ψ$. By induction, $\Ib \vDash ψ'$. It follows that $\Ib \vDash (ψ' \imp F) \imp F \equiv φ \imp F$.
 \end{itemize}
\end{proof}
\begin{theorem}
 $φ$ is tautological iff $φ'$ is tautological.
\end{theorem}
\begin{proof}
If $φ$ is tautological, then $\I \vDash φ$ for every $\L$-interpretation $\I$. Since every $\Lf$-interpretation is an $\L$-interpretation extended with either $ F \mapsto \bot$ or $ F \mapsto \top$, we obtain from the lemma that $\I' \vDash φ'$ for every $\Lf$-interpretation $\I'$. Hence $φ'$ is tautological.

Conversely, if $φ$ is not tautological, there is an $\L$-interpretation $\I$ such that $\I \not \vDash φ$. By the lemma, $\Ib \vDash φ' \imp F$ and since $\Ib(F) = \bot$, this means $\Ib \not \vDash φ'$. Hence $φ'$ is not tautological.
 \end{proof}
 
 We have thus shown that for each instance of the tautology problem of propositional logic, we can construct an equivalent instance of the tautology problem for the positive fragment of propositional logic. This construction is obviously possible in polynomial time. In other words, we have reduced the first problem to the second. Since the first problem is coNP-complete, so is the second.

\end{document}

\newtheorem{theorem}{Theorem}
\newtheorem{lemma}[theorem]{Lemma}

\theoremstyle{definition}

\newtheorem{definition}[theorem]{Definition}
