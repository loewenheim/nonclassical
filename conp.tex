\documentclass[a4paper,10pt]{article}
\usepackage[utf8]{inputenc}
\usepackage{uniinput}
\usepackage{amsthm}
\usepackage{amsmath}
\usepackage{amssymb}
\usepackage{fullpage}

\newcommand{\imp}{\supset}
\newcommand{\I}{\mathcal{I}}
\renewcommand{\L}{\mathcal{L}}


\newtheorem{theorem}{Theorem}
\newtheorem{lemma}[theorem]{Lemma}

\theoremstyle{definition}

\newtheorem{definition}[theorem]{Definition}

\begin{document}
\section{coNP-completeness of TAUT}

Let $\L$ be a fixed propositional language, i.e. a set of propositional atoms, that does not contain the symbol $F$.
\begin{definition}
 Let $φ$ a propositional formula in the language $\L$. We define a translation $'$ into the positive fragment of propositional logic:
 \begin{enumerate}
  \item For $φ \equiv A$: φ' $\equiv A$.
  \item For $φ \equiv \bot$: $φ' \equiv F$.
  \item For $\circ \in \{∨,∧,\imp\}$ and $φ \equiv ψ_1 \circ ψ_2$: $φ' \equiv ψ_1' \circ ψ_2'$.
  \item For $φ \equiv ¬ψ$: $φ' \equiv ψ' \imp F$.
 \end{enumerate}
\end{definition}

Clearly, the range of $'$ contains no instances of $¬$ or $\bot$.

If $\I$ is an interpretation of $\L$, we can also view it as an interpretation of $L \cup \{F\}$ by letting $\I(F) = \bot$.
\begin{lemma}
 Let $\I$ be an interpretation of $\L$ and $φ$ an $\L$-formula.
 \begin{enumerate}
  \item If $\I \vDash φ$ then $\I \vDash φ'$.
  \item If $\I \not\vDash φ$ then $\I \vDash φ \imp F$.
 \end{enumerate}
\end{lemma}
\begin{proof}
 By induction.
 \begin{itemize}
  \item $φ \equiv A$: trivial.
  \item $φ \equiv \bot$: trivial.
  \item $φ \equiv ψ_1 ∧ ψ_2$: Assume $\I \vDash φ$. Then $\I \vDash ψ_1$ and $\I \vDash ψ_2$. By induction, $\I \vDash ψ_1'$ and $\I \vDash ψ_2'$ and hence $\I \vDash ψ_1' ∧ ψ_2 \equiv φ'$.
  
  Assume $\I \not \vDash φ$. Then $\I \not \vDash ψ_1$ or $\I \not \vDash ψ_2$. Say $\I \not \vDash ψ_1$ w.l.o.g. Then by induction, $\I \vDash ψ_1' \imp F$ and hence $\I \not \vDash (ψ_1' ∧ ψ_2') \imp F \equiv φ' \imp F$.
  \item $φ \equiv ψ_1 ∨ ψ_2, φ \equiv ψ_1 \imp ψ_2$: Analogous to the previous case.
  \item $φ \equiv ¬ ψ$: Assume $\I \vDash φ$, then $\I \not \vDash ψ$. By induction $\I \vDash ψ' \imp F \equiv φ'$.
  
  On the other hand, if $\I \not \vDash φ$, then $\I \vDash ψ$. By induction, $\I \vDash ψ'$. It follows that $\I vDash (ψ' \imp F) \imp F \equiv φ \imp F$.
 \end{itemize}
\end{proof}
\begin{theorem}
 $φ$ is tautological iff $φ'$ is tautological.
\end{theorem}
\begin{proof}
 By the lemma, if $φ$ is satisfiable, then so is $φ' ∧ ¬F$ ($\I$ satisfies it). By contraposition, if $φ' \imp F$ is tautological, $φ$ is unsatisfiable.
\end{proof}

\end{document}

\newtheorem{theorem}{Theorem}
\newtheorem{lemma}[theorem]{Lemma}

\theoremstyle{definition}

\newtheorem{definition}[theorem]{Definition}
