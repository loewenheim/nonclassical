\documentclass[a4paper,10pt]{article}
\usepackage[utf8]{inputenc}
\usepackage{uniinput}
\usepackage{amsthm}
\usepackage{amsmath}
\usepackage{amssymb}
\usepackage{fullpage}
\usepackage{gslist}
\usepackage{gsgraph}

\newcommand{\imp}{\supset}
\newcommand{\I}{\mathcal{I}}
\newcommand{\F}{\mathcal{F}}
\newcommand{\Ib}{\I_{¬ F}}
\newcommand{\It}{\I_{F}}
\renewcommand{\L}{\mathcal{L}}
\newcommand{\Lf}{\L_F}
\newcommand{\Dia}{\Diamond}
\newcommand{\M}{\mathcal{M}}


\newtheorem{theorem}{Theorem}
\newtheorem{lemma}[theorem]{Lemma}
\newtheorem*{problem*}{Problem}

\theoremstyle{definition}

\newtheorem{definition}[theorem]{Definition}

\title{NCL exercises 2}
\author{Sebastian Zivota}

\begin{document}
\maketitle

\section*{Exercise 12}

\begin{problem*}
For each of the following formulas:
Find a Kripke model in which the formula is true in some world. Is
there a Kripke model for the formula?  If not, why?
(Try to use as few worlds as possible.)

\begin{enumerate}
 \item $\Dia p ∧ \Dia \Box p ∧ ¬ \Box p$
 \item $p ∧ \Box p ∧ ¬ \Dia p$
 \item $(p \imp q) ∧ \Dia (p ∧ ¬q)$
 \item $¬ p ∧ \Dia \Dia p ∧ ¬ \Box\Dia p ∧ \Dia\Box ¬p$
 \item $\Dia p ∧ (\Box p \imp \Box\Dia p) ∧ \Box (p \imp ¬\Dia p)$
\end{enumerate}
\end{problem*}
\begin{enumerate}
 \item $φ \equiv \Dia p ∧ \Dia \Box p ∧ ¬ \Box p$
 
 Let $\M = \vcenter{\begin{graph}(0.5,1)(-0.2,-0.5)
\pos  0=(0,0);1=(1,0).
%\style edgelabel{close}.
\style node{plain};nodelabel{0};edgelabel{W}.
\node 0{$w$};1{$v$}.
\style nodelabel{SE}.
\node 1{$p$}.
\edge 0-0-1.
\end{graph}}$. Then $(\M, w)\models φ$.
 
 Let $\M$ be any Kripke structure and assume $\M \models φ$. Then both $\Dia \Box p$ and $¬ \Box p$ must hold in every node of the $\M$. But if $(\M, w) \models \Dia \Box p$ there is a $w' \in M$ such that $w R w'$ and $(\M, w') \models \Box p$. This contradicts $(\M, w') \models ¬\Box p$.
 
\item $p ∧ \Box p ∧ ¬ \Dia p$

The structure with a single world in which $p$ holds and no edges satisfies this formula (in its single world and therefore as a whole).

\item $φ \equiv (p \imp q) ∧ \Dia (p ∧ ¬q)$

Let $\M = \vcenter{\begin{graph}(0.5,1)(-0.2,-0.5)
\pos  0=(0,0);1=(1,0).
%\style edgelabel{close}.
\style node{plain};nodelabel{0};edgelabel{W}.
\node 0{$w$};1{$v$}.
\style nodelabel{SE}.
\node 0{$p,q$};1{$p$}.
\edge 0-1.
\end{graph}}$. Then $(\M, w)\models φ$.

$φ$ can be written as $ψ ∧ \Dia ¬ ψ$. Similarly to 1), it is unsatisfiable on the level of structures: On the one hand, $ψ$ must hold in every world, but on the other hand, $¬ψ$ must hold in at least one.

\item $φ \equiv ¬ p ∧ \Dia \Dia p ∧ ¬ \Box\Dia p ∧ \Dia\Box ¬p$

$φ$ is equivalent to $¬p ∧ \Dia \Dia p ∧ \Dia ¬ \Dia p$.

 
 Let $\M = \vcenter{\begin{graph}(0.5,1)(-0.2,-0.5)
\pos  0=(0,0);1=(1,0);2=(2,0).
%\style edgelabel{close}.
\style node{plain};nodelabel{0};edgelabel{W}.
\node 0{$w$};1{$v$}.
\style nodelabel{SE}.
\node 1{$p$}.
\edge 0-0-1.
\end{graph}}$. Then $(\M, w)\models φ$: $\Dia \Dia p$ holds at $w$ because $p$ holds at $v$; $\Dia ¬ \Dia p$ holds at $w$ because $v$ has no successors.

$φ$ cannot have a model because $¬p$ would have to hold in every world and $p$ would have to hold in some worlds.

 \item $φ \equiv \Dia p ∧ (\Box p \imp \Box\Dia p) ∧ \Box (p \imp ¬\Dia p)$

 Let $\M = \vcenter{\begin{graph}(0.5,1)(-0.2,-0.5)
\pos  0=(0,0);1=(1,0);2=(2,0).
%\style edgelabel{close}.
\style node{plain};nodelabel{0};edgelabel{W}.
\node 0{$w$};1{$v$}.
\style nodelabel{SE}.
\node 1{$p$}.
\edge 0-0-1.
\end{graph}}$. Then $(\M, w)\models φ$: $\Dia p$ is true at $w$ because $p$ is true at $v$; $\Box p \imp \Box\Dia p$ is true because the premise is false; $\Box (p \imp ¬\Dia p)$ is true because the only successor world of $w$ that satisfies $p$ (namely $v$) also satisfies $¬\Dia p$.

$φ$ is unsatisfiable: Assume $\M \models φ$ and $w \in M$. Since $(\M, w) \models \Dia p$, there is a $w' \in M$ such that $w R w'$ and $(\M, w') \models p$. But since $(\M, w) \models \Box (p \imp ¬\Dia p)$, it follows that $(\M, w') \models ¬ Dia p$ and hence $\Dia p$ is not satisfied at every world.
\end{enumerate}

\section*{Exercise 13}
\begin{problem*}
Consider the graph $\F$:
\begin{align*}
\begin{graph}(2,0.5)(0,0)
\pos  0=(0,0);1=(1,0).
%\style edgelabel{close}.
\style node{plain};nodelabel{0}.
\node 0{w};1{u}.
\edge 0-0-1-1-0.
\end{graph}
\end{align*}
Find  3--5  further  examples  of  modal  formulas  with  one  schematic variable  that  are  valid  in $\F$, above, such that removal  of  some (which?)  accessibilities leads to invalidity.
\end{problem*}

\begin{enumerate}
 \item $\Box A \imp \Dia A$ is valid, but becomes invalid if, for instance, both edges originating from $w$ are removed: then $\Box A$ is vacuously true at $w$, but $\Dia A$ is false.
 \item Similarly, if $φ$ is any tautology, $\Dia φ$ is valid in $\F$ but becomes invalid if all edges originating from one of the nodes are removed.
 
\end{enumerate}

\end{document}