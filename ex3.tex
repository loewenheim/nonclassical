\documentclass[a4paper,10pt]{article}
\usepackage[utf8]{inputenc}
\usepackage{uniinput}
\usepackage{amsthm}
\usepackage{amsmath}
\usepackage{amssymb}
\usepackage{fullpage}
\usepackage{gslist}
\usepackage{gsgraph}

\newcommand{\imp}{\supset}
\newcommand{\I}{\mathcal{I}}
\newcommand{\K}{\textbf{K}}
\newcommand{\F}{\mathcal{F}}
\newcommand{\Ib}{\I_{¬ F}}
\newcommand{\It}{\I_{F}}
\renewcommand{\L}{\mathcal{L}}
\newcommand{\Lf}{\L_F}
\newcommand{\Dia}{\Diamond}
\newcommand{\M}{\mathcal{M}}
\newcommand{\dom}{\mathrm{dom}}


\newtheorem{theorem}{Theorem}
\newtheorem{lemma}[theorem]{Lemma}
\newtheorem*{problem*}{Problem}

\theoremstyle{definition}

\newtheorem{definition}[theorem]{Definition}

\title{NCL exercises 3}
\author{Sebastian Zivota}

\begin{document}
\maketitle

\section*{Exercise 14}

\begin{problem*}
Show that the intersection of two logics is also a logic.
What about unions of logics?

\end{problem*}
Let $\L_1, \L_2$ be logics and $\L = \L_1 ∪ \L_2$. If $F \in \L$, then $F \in \L_i$ and hence $Fσ \in \L_i$ for any substitution $σ$ and $i = 1,2$. Therefore $F \in \L$.\\
Let $F, F \imp G \in \L$. Then $F, F \imp G \in \L_i$ and hence $G \in \L_i$ because $\L_i$ are logics. It follows that $G \in \L$.

The union of logics need not be a logic: Let $\L_1 = \{ϕ ∧ ψ \}, \L_2 = \{(ϕ ∧ ψ) \imp ¬ρ \}$, where $ϕ, ψ, ρ$ are arbitrary formulas. They are both trivially closed under substitution and modus ponens ($\L_1$ contains no implications, while $\L_2$ contains only implications with conjunctions as premises). Since $A ∧ B \in \L_1$ and $(A ∧ B) \imp ¬ C \in \L_2$, if their union were a logic, it would have to contain $¬C$. But since neither logic contains negations, this is not the case.
\section*{Exercise 16}
\begin{problem*}
The set of formulas that are valid in a particular Kripke frame do always form a logic extending CL. Prove this fact.
\end{problem*}
Let $\F$ be a frame and $\L$ the set of formulas valid in $\F$, i.e. valid in every model based on $\F$.

We have to show that $\L$ is closed under substitutions. Let $F \in \L$, $σ$ a substitution and $\M$ an arbitrary model on $\F$. We have to show that $\M \models Fσ$. Let $\{A_1,…A_n\} = \dom σ ∩ PV(F)$ and $G_i = σ(A_i)$ for $i = 1,…,n$. Let $\M'$ be the model on $\F$ such that $ν_{\M'}(A_i, w) = ν_{\M}(G_i, w)$ for $i = 1,…,n$ and $ν_{\M'}(B, w) = ν_{\M}(B,w)$ for all other propositional variables $B$. By the assumption, $\M' \models F$. It is easy to see that this implies $\M \models Fσ$.

As for MP, let $F, F \imp G \in \L$ and fix a structure $\M$ on $\F$ and a world $w \in \M$. Clearly, $(\M, w) \models F$ and $(\M, w) \models F \imp G$. By the propositional semantics of $\imp$, $(\M, w) \models G$ must hold as well. Since $\M$ and $w$ were arbitrary, it follows that $G \in \L$.

\section{Exercise 19}
\begin{problem*}
 Prove these facts:
 \begin{enumerate}
  \item $\M \models F \Rightarrow \M \models \Box F$
  \item $\F \models F \Rightarrow \F \models \Box F$
 \end{enumerate}
\end{problem*}

\begin{enumerate}
 \item Let $\M \models F$ and $w$ a world in $\M$. Since $\M \models F$, we know that $(\M, w') \models F$ for each $w' \in \M$ and hence certainly for each $w'$ that is reachable from $w$. In other words, $(\M, w) \models \Box F$ and, since $w$ was arbitrary, $\M \models \Box F$.
 \item Let $\F \models F$ and $\M$ an arbitrary structure on $\F$. Since $\M \models F$, by 1) we obtain $\M \models \Box F$. Since $\M$ was arbitrary, this entails $\F \models \Box F$.
\end{enumerate}


\section{Exercise 20}
\begin{problem*}
 Prove the soundness of $\K$ (i.e. $\K \subseteq \L := \{A \ |\  \forall F: \F \models A\}$) using mentioned facts.
\end{problem*}

We need to show that
\begin{enumerate}
 \item all propositional tautologies are in $\L$;
 \item the schema $(K) := \Box(A \imp B) \imp (\Box A \imp \Box B)$ is in $\L$;
 \item $\L$ is closed under MP and necessitation.
\end{enumerate}

Proof:
\begin{enumerate}
 \item Any propositional tautology holds in every possible propositional structure and hence in every world of every model on every frame.
 \item Let $\F$ an arbitrary frame, $\M$ a structure on $\F$ and $w \in \M$. We need to show that $(\M, w) \models (K)$. Suppose $(\M, w) \models \Box(A\imp B)$. This means that for all $w'$ accessible from $w$, $(\M, w') \models A \imp B$. If $(\M, w) \models \Box A$, then for each such $w'$, $(\M, w') \models A$ as well. It follows that $(\M, w') \models B$ and hence $(\M, w) \models \Box B$.
 \item These follow from exercises 16 and 19, respectively.
\end{enumerate}


\end{document}