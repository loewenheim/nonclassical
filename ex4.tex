\documentclass[a4paper,10pt]{article}
\usepackage[utf8]{inputenc}
\usepackage{uniinput}
\usepackage{amsthm}
\usepackage{amsmath}
\usepackage{amssymb}
\usepackage{fullpage}
\usepackage{gslist}
\usepackage{gsgraph}

\newcommand{\imp}{\supset}
\newcommand{\I}{\mathcal{I}}
\newcommand{\K}{\textbf{K}}
\renewcommand{\S}[1]{\textbf{S#1}}
\newcommand{\F}{\mathcal{F}}
\newcommand{\Ib}{\I_{¬ F}}
\newcommand{\It}{\I_{F}}
\renewcommand{\L}{\mathcal{L}}
\newcommand{\Lf}{\L_F}
\newcommand{\Dia}{\Diamond}
\newcommand{\M}{\mathcal{M}}


\newtheorem{theorem}{Theorem}
\newtheorem{lemma}[theorem]{Lemma}
\newtheorem*{problem*}{Problem}

\theoremstyle{definition}

\newtheorem{definition}[theorem]{Definition}

\title{NCL exercises 4}
\author{Sebastian Zivota}

\begin{document}
\maketitle

\section*{Exercise 31}

\begin{problem*}
Explain why $\S5$ is sound and complete for frames where accessibility is an equivalence relation, using the following theorem:

\begin{theorem}
 The normal modal logic with axioms $\{A_{i_1},…,A_{i_n}\} \subseteq \{A_1,…,A_{10}\}$ is sound and complete for frames where the accessibility relation satisfies the corresponding properties $\{E_{i_1},…,E_{i_n}\}$.
\end{theorem}
\end{problem*}
By the theorem, \S5 is sound and complete for reflexive Euclidean frames. Thus, we need to show that a relation is an equivalence iff it is reflexive and Euclidean.

Assume $R$ is reflexive and Euclidean and let $u {R}v$. Because of reflexivity, $u {R} u$ and together with the Euclidean property we obtain $v {R} u$, so $R$ is symmetric. Now let $u {R} v$ and $v {R} w$. We already know $R$ to be symmetric, so $v {R} u$. From $v {R} u$ and $v {R} w$ we deduce $u {R} w$, so $R$ is transitive.

Now assume $R$ is an equivalence relation and let $u {R} v$ and $u {R} w$. Because of symmetry we have $v {R} w$ and because of transitivity we obtain $v {R} w$.
\section*{Exercise 32}
\begin{problem*}
Prove: For every frame $\F$, $R_1 \subseteq R_2$ iff $\Box_2 A \imp \Box_1 A$ is valid in $\F$.
\end{problem*}
Assume $R_1 \subseteq R_2$. Let $\M$ be a model on $\F$ and $w$ a world in $\M$. Assume $(\M, w) \models \Box_2 A$. We need to show $(\M, w) \models \Box_1 A$, which means that for every $w'$ with $w {R_1} w'$, $(\M, w') \models A$. Since $R_1 \subseteq R_2$, $w {R_2} w'$ and hence $(\M, w) \models \Box_1 A$.

Now assume $R_1 \not \subseteq R_2$, i.e. there are $w, w' \in \F$ with $(w,w') \in R_1\setminus R_2$. Construct a model $\M$ on $\F$ as follows:
\begin{align*}
ν_{\M}(A, v) =\begin{cases}
	      0, &v = w',\\
	      1, &\text{otherwise.}
              \end{cases}
\end{align*}
Note that by this definition, $ν_{\M}(A,v) = 1$ for all $v$ such that $w{R_2}v$. Now we have $(\M, w) \models \Box_2 A$ and $(\M, w) \not \models \Box_1 A$, so $\Box_2 A \imp \Box_1 A$ is not valid in $\M$ and hence not in $\F$ either.
\end{document}